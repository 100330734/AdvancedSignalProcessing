% --------------------------------------------------------------
% This is all preamble stuff that you don't have to worry about.
% Head down to where it says "Start here"
% --------------------------------------------------------------

\documentclass[12pt]{article}


\usepackage[utf8]{inputenc}
\usepackage[margin=1in]{geometry} 
\usepackage{amsmath,amsthm,amssymb}
\usepackage[backend=biber, style=ieee, isbn=false,sortcites, maxbibnames=5, minbibnames=1]{biblatex}
\usepackage{hyperref}
\usepackage{xcolor}
\usepackage{amsmath}
\usepackage{graphicx}
\usepackage{caption,subcaption}
\captionsetup{labelsep = period} % Figure 2. Caption (rather than Figure 2: Caption)

\usepackage[]{hyperref}
\hypersetup{
	colorlinks   = true, %Colours links instead of ugly boxes
	urlcolor     = blue, %Colour for external hyperlinks
	linkcolor    = red, %Colour of internal links
	citecolor   = green %Colour of citations
}
\addbibresource{bibliography.bib}


\newenvironment{theorem}[2][Theorem]{\begin{trivlist}
		\item[\hskip \labelsep {\bfseries #1}\hskip \labelsep {\bfseries #2.}]}{\end{trivlist}}
\newenvironment{lemma}[2][Lemma]{\begin{trivlist}
		\item[\hskip \labelsep {\bfseries #1}\hskip \labelsep {\bfseries #2.}]}{\end{trivlist}}
\newenvironment{exercise}[2][Exercise]{\begin{trivlist}
		\item[\hskip \labelsep {\bfseries #1}\hskip \labelsep {\bfseries #2.}]}{\end{trivlist}}
\newenvironment{reflection}[2][Reflection]{\begin{trivlist}
		\item[\hskip \labelsep {\bfseries #1}\hskip \labelsep {\bfseries #2.}]}{\end{trivlist}}
\newenvironment{proposition}[2][Proposition]{\begin{trivlist}
		\item[\hskip \labelsep {\bfseries #1}\hskip \labelsep {\bfseries #2.}]}{\end{trivlist}}
\newenvironment{corollary}[2][Corollary]{\begin{trivlist}
		\item[\hskip \labelsep {\bfseries #1}\hskip \labelsep {\bfseries #2.}]}{\end{trivlist}}

\newcommand{\indicator}{\mathbb{I}} 
\newcommand{\expectation}{\mathbb{E}} 
\newcommand{\bs}{\boldsymbol}

\DeclareMathOperator*{\argmax}{arg\,max}
\DeclareMathOperator*{\argmin}{arg\,min}

\begin{document}
	
% --------------------------------------------------------------
%                         Start here
% --------------------------------------------------------------

%\renewcommand{\qedsymbol}{\filledbox}

\title{Assignment 2: HMM for Categorical Data Sequences}%
\author{Daniel Barrejón Moreno} %if necessary, replace with your course title

\maketitle
\section{Complete data log-likehood $l_{c}\left(\boldsymbol\theta\right)$ for the N sequences}
The expression of the complete data log-likelihood will have the following form
\begin{equation}
\label{ll1}
l_{c}\left(\boldsymbol\theta\right) = \log p ( S , Y | \boldsymbol\theta ) = \log \prod _ { n = 1 } ^ { N } \left( p \left( s _ { 1 } ^ { n } | \boldsymbol\pi \right) \prod _ { t = 2 } ^ { T _ { n } } p \left( s _ { t } ^ { n } | s _ { t - 1 } ^ { n } , \mathbf { A } \right) \right) \left( \prod _ { t = 1 } ^ { T _ { n } } p \left( \mathbf { y } _ { t } ^ { n } | s _ { t } ^ { n } , \mathbf { B } \right) \right),
\end{equation}
where $\log$ operator is the Naperian logarithm, $S$ represents the hidden states of the model, $Y$ is the observed continuous sequence, $\mathbf { A }$ stands for the state transition probabilities,$B$ the observatoin emission probabilities and $\boldsymbol\pi$ is the initial state probability distribution.\\

\noindent The parameters of the model are
\begin{equation}
\boldsymbol\theta = \{ \mathbf { A } , \mathbf { B } , \boldsymbol\pi \},
\end{equation}
where $\mathbf { A }$  is the state transition matrix, $\mathbf { B } $ is the emission matrix and  $\boldsymbol\pi $ represents the probability of each state.
Equation \ref{ll1} is composed by three main terms: the $\boldsymbol\pi$, $\mathbf { A }$ and $\mathbf { B } $ ones which can be rewritten as
\begin{align}
p \left( s _ { 1 } ^ { n } | \boldsymbol\pi \right) &= \prod\limits_{k=1}^{K}\pi_{k}^{\indicator \lbrace s_{1}^{n}=k|Y,\boldsymbol\theta\rbrace},\\
p \left( s _ { t } ^ { n } | s _ { t - 1 } ^ { n } , \mathbf { A } \right) &= \prod\limits_{k=1}^{K} \prod\limits_{k'=1}^{K}a_{k,k'}^{\indicator \lbrace s_{t-1}^{n}=k,s_{t}^{n}=k'|Y,\boldsymbol\theta\rbrace},\\
p \left( \mathbf { y } _ { t } ^ { n } | s _ { t } ^ { n } , \mathbf { B } \right) &= \prod\limits_{k=1}^{K} p \left( \mathbf { y } _ { t } ^ { n } | \boldsymbol b_{k} \right) = \prod\limits_{k=1}^{K} p \left( \mathbf { y } _ { t } ^ { n } | \boldsymbol\theta_{k} \right) \label{p3LL}.
\end{align}
In this case, $\indicator$ represents an indicator function, $k = \lbrace{1,...,K}\rbrace$ the current latent state of the model, $a_{k,k'}$ the $k$th row and $k'$th column element of the forementioned matrix $\mathbf { A }$ and $t = \lbrace{1,...,T_{n}}\rbrace$ the position of the state in sequence $n$. Please notice that now, $\mathbf{b}_{k}$ (that belonged to $\mathbf{B}$) becomes $\boldsymbol\theta_{k}$, which denotes the hyperparameters of the categorical distribution that the data follow.\\

\noindent Since our data follow a categorical distribution, Equation \ref{p3LL} can be expressed as
\begin{equation}
p \left( \mathbf { y } _ { t } ^ { n } | \boldsymbol\theta_{k} \right) = \prod\limits_{j=1}^{Dt} \text{Cat}(y_{j,t}|\boldsymbol\theta_{k}).
\end{equation}
With a further development we can get
\begin{equation}
p \left( \mathbf { y } _ { t } ^ { n } | \boldsymbol\theta_{k} \right) = \prod\limits_{j=1}^{Dt} \text{Cat}(y_{j,t}|\boldsymbol\theta_{k}) = \prod\limits_{m=1}^{I}\prod\limits_{j=1}^{Dt}\theta_{k,m}^{\indicator\lbrace{y_{j,t}^{n}=m}\rbrace} = \prod\limits_{m=1}^{I}\theta_{k,m}^{\sum\limits_{j=1}^{Dt}\indicator\lbrace{y_{j,t}^{n}=m}\rbrace} = \prod\limits_{m=1}^{I}\theta_{k,m}^{\mu^{n}_{t,m}},
\end{equation}
being $\mu^{n}_{t,m}$
\begin{equation}
\mu^{n}_{t,m} = \sum\limits_{j=1}^{Dt}\indicator\lbrace{y_{j,t}^{n}=m}\rbrace.
\end{equation}
The final expression of the complete data log-likelihood can be expressed in the following form	
\begin{equation}
\begin{split}
l_{c}\left(\boldsymbol\theta\right) &= \sum\limits_{n=1}^{N}\sum\limits_{k=1}^{K}\indicator \lbrace s_{1}^{n}=k|Y,\boldsymbol\theta\rbrace \log(\pi_{k}) + \\
&+ \sum\limits_{n=1}^{N}\sum\limits_{k=1}^{K}\sum\limits_{k'=1}^{K}\sum\limits_{t=1}^{T_{n}}\indicator \lbrace s_{t-1}^{n}=k,s_{t}^{n}=k'|Y,\boldsymbol\theta\rbrace \log(a_{k,k'}) + \\
&+ \sum\limits_{n=1}^{N}\sum\limits_{t=1}^{T_{n}}\indicator \lbrace s_{t}^{n}=k|Y,\boldsymbol\theta\rbrace \sum\limits_{m=1}^{I}\mu^{n}_{t,m} \log(\theta_{k,m}).
\end{split}
\end{equation}
\section{Expected Complete Data Logl-likelihood $Q \left( \boldsymbol { \theta } , \boldsymbol { \theta } ^ { t - 1 } \right)$}
The expected complete data log-likelihood has the following form
\begin{equation}
Q \left( \boldsymbol { \theta } , \boldsymbol { \theta } ^ { t - 1 } \right) = E \left\{ l _ { c } ( \boldsymbol { \theta } ) | \mathcal { D } , \boldsymbol { \theta } ^ { t - 1 } \right\}
\end{equation}
As in the previous section, and considering the nature of $l_{c}$ and its three main components, this expectation calculation can be divided in three.
\begin{align}
& \expectation \left( \sum _ { n = 1 } ^ { N } \indicator \left( s _ { 1 } ^ { n } = k | Y , \bs\theta \right) \right) = \sum _ { n = 1 } ^ { N } \gamma _ { n , 1 } ( k ), \\
& \expectation \left( \sum _ { n = 1 } ^ { N } \sum _ { t = 2 } ^ { T _ { n } } \indicator \left( s _ { t - 1 } ^ { n } = k , s _ { t } ^ { n } = k' | Y , \bs\theta \right) \right) = \sum _ { n = 1 } ^ { N } \sum _ { t = 2 } ^ { T _ { n } } \xi _ { n , t } ( k , k' ), \\
& \expectation \left( \sum _ { n = 1 } ^ { N } \sum _ { t = 1 } ^ { T _ { n } } \indicator \left( s _ { t } ^ { n } = k | Y , \bs\theta \right) \right) = \sum _ { n = 1 } ^ { N } \sum _ { t = 1 } ^ { T _ { n } } \gamma _ { n , t } ( k ).
\end{align}
Where 
\begin{equation}
\sum _ { k = 1 } ^ { K } \gamma _ { n , t } ( k ) = 1.
\end{equation}
\noindent Being $\xi _ { n,t } ( k , k' )$ 
\begin{equation}
\xi _ { n,t } ( k , k' ) = \alpha _ { t-1 }^{n} ( k ) a _ { k,k' } \prod\limits_{m=1}^{I} \theta_{k,m}^{\mu^{n}_{t,m}} \beta _ { t }^{n} ( k' ),
\end{equation}
and $ \gamma _ { n , t } ( k ) $
\begin{equation}\label{eq:gamma}
\gamma _ { n , t } ( k )\propto \beta _ { t }^{n} ( k )\alpha _ { t }^{n} ( k )
\end{equation}
The terms $\alpha$ and $\beta$ are computed by means of the \textbf{forward-backward algorithm }as follows
\begin{align}
\label{alfabeta}
&\alpha _ { 1 }^{n} ( k ) = \pi _ { k } \prod\limits_{m=1}^{I}\theta_{k,m}^{\mu^{n}_{1,m}}, \\
&\alpha _ { t }^{n} ( k ) = \left( \sum _ { k' = 1 } ^ { K } \alpha _ { t - 1 }^{n} ( k' ) a _ { k',k } \right) \prod\limits_{m=1}^{I}\theta_{k,m}^{\mu^{n}_{t,m}}, \\
&\beta _ { T_{n} }^{n} ( k ) = 1, \\
& \beta _ { t }^{n} ( k ) = \sum _ {k' = 1 } ^ { K } a _ { k,k'}  \prod\limits_{m=1}^{I}\theta_{k',m}^{\mu^{n}_{t+1,m}}  \beta _ { t + 1 }^{n} ( k' ).
\end{align}
So the complete expression for $Q \left( \boldsymbol { \theta } , \boldsymbol { \theta } ^ { t - 1 } \right)$ is now
\begin{equation}
\begin{split}
Q \left( \boldsymbol { \theta } , \boldsymbol { \theta } ^ { t - 1 } \right) &= \sum\limits_{k=1}^{K}\sum _ { n = 1 } ^ { N } \gamma _ { n , 1 } ( k ) \log(\pi_{k}) + \\
&+ \sum\limits_{k=1}^{K}\sum\limits_{k'=1}^{K}\sum _ { n = 1 } ^ { N } \sum _ { t = 2 } ^ { T _ { n } } \xi _ { n , t } ( k , k' ) \log(a_{k,k'}) + \\
&+ \sum\limits_{k=1}^{K}\sum\limits_{m=1}^{I}\sum _ { n = 1 } ^ { N } \sum _ { t = 1 } ^ { T _ { n } } \gamma _ { n , t } ( k )\mu^{n}_{t,m} \log(\theta_{k,m}).
\end{split}
\end{equation}
\section{ML Inference}
There are three parameters for the model, which are $\mathbf{A}$, $\bs\theta$ and $\bs\pi$, and they are computed by means of Lagrange multipliers.
\subsection{ML estimation of $\pi_{k}$}
In order to compute $\pi_{k}$, we first need to take into account the following restrictions
\begin{align}
& 0 \leq \pi_{k} \leq 1 \\
& \sum\limits_{k = 1}^{K} \pi_{k} = 1.
\end{align}
Now, let us define the lagrangian as
\begin{equation}
\label{lagrange_pik}
L\left( Q(\pi_{k}),\lambda \right) = Q(\pi_{k}) + \lambda \left( \sum \limits_{k=1}^{K} \pi_{k} - 1 \right),
\end{equation}
which will be optimized in this way
\begin{equation}
\label{minmax_lagrange_pik}
\min_{\substack{\lambda}} \max_{\substack{\pi_{k}}} \lbrace L\left( Q(\pi_{k}),\lambda \right) \rbrace.
\end{equation}
By first taking the derivative with respect to $\pi_{k}$ and equating it to 0
\begin{equation}
\label{deriv_pik}
\begin{split}
& \dfrac{\partial L}{\partial \pi_{k}} = \sum \limits_{n=1}^{N} \dfrac{\gamma_{n,1}(k)}{\pi_{k}} - \lambda = 0, \\
& \pi_{k} = \dfrac{1}{\lambda} \sum \limits_{n=1}^{N} \gamma_{n,1}(k).
\end{split}
\end{equation}
And later with respect to $\lambda$
\begin{equation}
\begin{split}
& \dfrac{\partial L}{\partial \lambda} = \sum \limits_{k=1}^{K} \pi_{k} - 1 = 0,\\
& \sum \limits_{k=1}^{K} \pi_{k} = 1.\\
\end{split}
\end{equation}
Taking into account the previous equation, if in both sides of Equation \ref{deriv_pik} summatories all over $K$ are taken, then the value of $\lambda$ can be obtained
\begin{equation}
\begin{split}
& \sum \limits_{k=1}^{K}\pi_{k} = \sum \limits_{k=1}^{K} \dfrac{1}{\lambda} \sum \limits_{n=1}^{N} \gamma_{n,1}(k)\\
& 1 = \dfrac{1}{\lambda} \sum \limits_{n=1}^{N}\sum \limits_{k=1}^{K}\gamma_{n,1}(k).\\
\end{split}
\end{equation}
Now, considering the previously imposed restrictions and recalling that
\begin{equation}
\sum _ { k = 1 } ^ { K } \gamma _ { n , t } ( k ) = 1,
\end{equation}
the value of $\lambda$ is
\begin{equation}
\begin{split}
& 1 = \dfrac{1}{\lambda} \sum \limits_{n=1}^{N}1.\\
& \lambda = N.\\
\end{split}
\end{equation}
And with that value of $\lambda$ the estimated value of $\pi_{k}$ is
\begin{equation}
\widehat{\pi}_{k} = \dfrac{1}{N} \sum \limits_{n=1}^{N} \gamma_{n,1}(k).
\end{equation}
\subsection{ML estimation of $\theta_{k,m}$}
Now, the parameter to be estimated is $\theta_{k,m}$ and the constraint is now
\begin{equation}
\label{constraint_thetakm}
\sum \limits_{m=1}^{I} \theta_{k,m} = 1.
\end{equation}
Being the whole expression
\begin{equation}
\label{lagrange_thetakm}
L\left( Q(\theta_{k,m}),\lambda \right) = Q(\theta_{k,m}) + \lambda \left( \sum \limits_{m=1}^{I} \theta_{k,m} - 1 \right),
\end{equation}
which will be optimized in this way
\begin{equation}
\label{minmax_lagrange_thetakm}
\min_{\substack{\lambda}} \max_{\substack{\theta_{k,m}}} \lbrace L\left( Q(\theta_{k,m}),\lambda \right) \rbrace.
\end{equation}
By first taking the derivative with respect to $\theta_{k,m}$ and equating it to 0
\begin{equation}
\label{deriv_thetakm}
\begin{split}
& \dfrac{\partial L}{\partial \theta_{k,m}} = \sum \limits_{n=1}^{N}\sum _ { t = 1 } ^ { T _ { n } } \dfrac{\gamma_{n,t}(k)\mu_{t,m}^{n}}{\theta_{k,m}} - \lambda  = 0 , \\
& \theta_{k,m} = \dfrac{1}{\lambda} \sum \limits_{n=1}^{N} \sum _ { t = 1 } ^ { T _ { n } }\gamma_{n,t}(k)\mu_{t,m}^{n}.
\end{split}
\end{equation}
And later with respect to $\lambda$
\begin{equation}
\begin{split}
& \dfrac{\partial L}{\partial \lambda} = \sum \limits_{m=1}^{I} \theta_{k,m} - 1 = 0,\\
& \sum \limits_{m=1}^{I} \theta_{k,m} = 1.\\
\end{split}
\end{equation}
Taking into account the previous equation, if in both sides of Equations \ref{deriv_thetakm} summatories all over $I$ are taken, then the value of $\lambda$ can be obtained
\begin{equation}
\begin{split}
& \sum \limits_{m=1}^{I}\theta_{k,m} = \sum \limits_{m=1}^{I} \dfrac{1}{\lambda} \sum \limits_{n=1}^{N}\sum _ { t = 1 } ^ { T _ { n } } \gamma_{n,t}(k)\mu_{t,m}^{n}\\
& 1 = \dfrac{1}{\lambda} \sum \limits_{n=1}^{N}\sum _ { t = 1 } ^ { T _ { n } }\sum \limits_{m=1}^{I}\gamma_{n,t}(k)\mu_{t,m}^{n},\\
& \lambda = \sum \limits_{n=1}^{N}\sum \limits_{ t = 1 } ^ { T _ { n }}\sum \limits_{m=1}^{I}\gamma_{n,t}(k)\mu_{t,m}^{n}.\\
\end{split}
\end{equation}
And with that value of $\lambda$ the estimated value of $\theta_{k,m}$ is
\begin{equation}
\widehat{\theta}_{k,m} = \dfrac{\sum \limits_{n=1}^{N} \sum \limits_ { t = 1 } ^ { T _ { n } }\gamma_{n,t}(k)\mu_{t,m}^{n}}{\sum \limits_{n=1}^{N}\sum \limits_ { t = 1 } ^ { T _ { n } }\sum \limits_{m=1}^{I}\gamma_{n,t}(k)\mu_{t,m}^{n}}.
\end{equation}
\subsection{ML estimation of $a_{k,k'}$}
At last, the parameter to be estimated is $a_{k,k'}$ and the constraint is now
\begin{equation}
\label{constraint_akk}
\sum \limits_{k'=1}^{K} a_{k,k'} = 1.
\end{equation}
Being the whole expression
\begin{equation}
\label{lagrange_akk}
L\left( Q(a_{k,k'}),\lambda \right) = Q(a_{k,k'}) + \lambda \left( \sum \limits_{k'=1}^{K} a_{k,k'} - 1 \right),
\end{equation}
which will be optimized in this way
\begin{equation}
\label{minmax_lagrange_akk}
\min_{\substack{\lambda}} \max_{\substack{a_{k,k'}}} \lbrace L\left( Q(a_{k,k'}),\lambda \right) \rbrace.
\end{equation}
By first taking the derivative with respect to $a_{k,k'}$ and equating it to 0
\begin{equation}
\label{deriv_akk}
\begin{split}
& \dfrac{\partial L}{\partial a_{k,k'}} = \sum \limits_{n=1}^{N}\sum _ { t = 2 } ^ { T _ { n } } \dfrac{\xi_{n,t}(kk')}{a_{k,k'}} - \lambda = 0, \\
& a_{k,k'} = \dfrac{1}{\lambda} \sum \limits_{n=1}^{N} \sum _ { t = 2 } ^ { T _ { n } }\xi_{n,t}(kk').
\end{split}
\end{equation}
And later with respect to $\lambda$
\begin{equation}
\begin{split}
& \dfrac{\partial L}{\partial \lambda} = \sum \limits_{m=1}^{I} a_{k,k'} - 1 = 0 ,\\
& \sum \limits_{k'=1}^{K} a_{k,k'} = 1.\\
\end{split}
\end{equation}
Taking into account the previous equation, if in both sides of \eqref{deriv_akk} summatories all over $I$ are taken, then the value of $\lambda$ can be obtained
\begin{equation}
\begin{split}
& \sum \limits_{k'=1}^{K}a_{k,k'} = \sum \limits_{k'=1}^{K} \dfrac{1}{\lambda} \sum \limits_{n=1}^{N}\sum _ { t = 1 } ^ { T _ { n } } \xi_{n,t}(k,k')\\
& 1 = \dfrac{1}{\lambda} \sum \limits_{n=1}^{N}\sum _ { t = 1 } ^ { T _ { n } }\sum \limits_{k'=1}^{K}\xi_{n,t}(k,k'),\\
& \lambda = \sum \limits_{n=1}^{N}\sum \limits_{ t = 1 } ^ { T _ { n }}\sum \limits_{k'=1}^{K}\xi_{n,t}(k,k').\\
\end{split}
\end{equation}
And with that value of $\lambda$ the estimated value of $a_{k,k'}$ is
\begin{equation}
\widehat{a}_{k,k'} = \dfrac{\sum \limits_{n=1}^{N} \sum \limits_ { t = 1 } ^ { T _ { n } }\xi_{n,t}(k,k')}{\sum \limits_{n=1}^{N}\sum \limits_ { t = 1 } ^ { T _ { n } }\sum \limits_{k'=1}^{K}\xi_{n,t}(k,k')}.
\end{equation}


\subsection{MAP decoding based on Forward-Backwards}
In order to determine the state of each document for the different sequences we obtain the most probable state for each document at each sequence from the parameter $\gamma _ { n , t } ( k )$ which is computed in the E-step of the Baum-Welch EM using Equation \ref{eq:gamma}.


\subsection{ML Viterbi decoding}
As a comparison for the MAP decoding comment above, we have also implemented a Viterbi decoder. The Viterbi decoder is a dynamic programming algorithm for finding the most likely sequence of hidden states result of a sequence of observed events. Since we are dealing with hidden and observed states with HMM, the algorithm suits perfectly to our problem. The Viterbi decoding algorithm looks as follows
\begin{equation}
\underset { S } { \operatorname { argmax } } p ( S , Y ) = \underset { i } { \operatorname { argmax } } \left\{ \underset { s _ { 1 : T - 1 } } { \operatorname { argmax } } p \left( s _ { T } = i , s _ { 1 : T - 1 } , Y \right) \right\}.
\end{equation}
In particular, we will use an iterative implementation of the Viterbi Algorithm. For such implementation we define two iterative steps described below.

\begin{itemize}
	\item \textbf{Forward step}: Computed with the following equations
	\begin{align}
	\delta_{1}(k)&=\pi_{k} \prod\limits_{m=1}^{I}\theta_{k,m}^{\mu^{n}_{1,m}} \quad 1 \leq k \leq K\\
	\delta_{t}(k)&=\prod\limits_{m=1}^{I}\theta_{k,m}^{\mu^{n}_{t,m}} \max _{k'} a_{k',k} \delta_{t-1}(k') \quad 1 \leq k \leq K , 1 \leq t \leq T \\
	\varphi_{t}(k)&=\underset{k'}{\operatorname{argmax}} \hspace{0.1cm}a_{k',k} \delta_{t-1}(k') \quad 1 \leq k \leq K , 1 \leq t \leq T \\
	\end{align}
	\item \textbf{Backwards step}: The state estimation is computed using the following Equations
	\begin{align}
	\hat{\mathbf{s}}_{T}&=\underset{\boldsymbol{k}}{\arg \max } \delta_{T}(k)\\
	\hat{s}_{t}&=\varphi_{t+1}\left(\hat{s}_{t+1}\right)  1 \leq t \leq T.
	\end{align} 
\end{itemize}

\section{Experiments}
For the experiments...\\

\noindent One important aspect about the implementation of the algorithm is that instead of working with the actual $\xi _ { n,t } ( k , k' )$ we work with a similar implementation used in \cite{murphy2012machine}, which consider working with the expected sufficient statistics for the transition matrix, for a given observation sequence. This may be rewritten as follows
\begin{equation}
\xi_\Sigma(k,k^{\prime}) = \sum\limits_{t=2}^{T} p(S(t) = k, S(t+1) = k^{\prime} | y(1:T)),
\end{equation}
where the subscript $\Sigma$ indicates the sum over $t$. Notice that, for a given sequence, this matrix is no longer a tensor but a matrix of dimension $K\times K$.

\end{document}
