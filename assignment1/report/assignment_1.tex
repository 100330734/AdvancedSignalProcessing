% --------------------------------------------------------------
% This is all preamble stuff that you don't have to worry about.
% Head down to where it says "Start here"
% --------------------------------------------------------------
 
\documentclass[12pt]{article}
 

\usepackage[utf8]{inputenc}
\usepackage{intmacros}
\usepackage[margin=1in]{geometry} 
\usepackage{amsmath,amsthm,amssymb}
\usepackage[backend=biber, style=ieee, isbn=false,sortcites, maxbibnames=5, minbibnames=1]{biblatex}
\usepackage{hyperref}
\usepackage{xcolor}
\usepackage{amsmath}
\usepackage{graphicx}

\usepackage[]{hyperref}
\hypersetup{
	colorlinks   = false, %Colours links instead of ugly boxes
	urlcolor     = blue, %Colour for external hyperlinks
	linkcolor    = red, %Colour of internal links
	citecolor   = green %Colour of citations
}
\addbibresource{bibliography.bib}


\newenvironment{theorem}[2][Theorem]{\begin{trivlist}
\item[\hskip \labelsep {\bfseries #1}\hskip \labelsep {\bfseries #2.}]}{\end{trivlist}}
\newenvironment{lemma}[2][Lemma]{\begin{trivlist}
\item[\hskip \labelsep {\bfseries #1}\hskip \labelsep {\bfseries #2.}]}{\end{trivlist}}
\newenvironment{exercise}[2][Exercise]{\begin{trivlist}
\item[\hskip \labelsep {\bfseries #1}\hskip \labelsep {\bfseries #2.}]}{\end{trivlist}}
\newenvironment{reflection}[2][Reflection]{\begin{trivlist}
\item[\hskip \labelsep {\bfseries #1}\hskip \labelsep {\bfseries #2.}]}{\end{trivlist}}
\newenvironment{proposition}[2][Proposition]{\begin{trivlist}
\item[\hskip \labelsep {\bfseries #1}\hskip \labelsep {\bfseries #2.}]}{\end{trivlist}}
\newenvironment{corollary}[2][Corollary]{\begin{trivlist}
\item[\hskip \labelsep {\bfseries #1}\hskip \labelsep {\bfseries #2.}]}{\end{trivlist}}
 
\begin{document}
 
% --------------------------------------------------------------
%                         Start here
% --------------------------------------------------------------
 
%\renewcommand{\qedsymbol}{\filledbox}
 
\title{Assignment 1: EM for Categorical Data Advanced Signal Processing}%
\author{Daniel Barrejón Moreno} %if necessary, replace with your course title
 
\maketitle
 
\section{Problem formulation}
 
\noindent In this project we want to study a set of documents with a model following a mixture of categorical distributions. For a document $\x$ belonging to a set of documents $\X$, the marginal li,kelihood is expressed as follows
\begin{equation}\label{eq:marginal_x_given_z}
     p(\x|\bstheta) = \sum_{k=1}^{K}\pi_k\prod_{j=1}^{D}\Cat(x_j|\bstheta_k),
\end{equation}
where $K$ is the number of mixtures (in our case the topics), $D$ is the number of words that appear in a certain document and $\Cat(x_j|\bstheta_k)$ is the categorical distribution with $I$ categories and parameter $\bstheta_k$ which represents the probability that a certain category appears. Bear in mind that the model parameters are defined as $\bstheta = \{\bstheta_k, \bspi\}$. Since $\bstheta_k = (\theta_{k,1},\dots, \theta_{k,m}, \dots, \theta_{k,I})$ is a vector with the probabilities of topic $k$ probability, it must satisfy two constraints:
\begin{align}\label{eq:theta_constraint}
& 0 < \theta_{k,m} < 1\\
\text{and} \quad & \sum\limits_{m=1}^{I}\theta_{k,m} = 1.
\end{align}
\noindent The expression for the log li,kelihood of the observed data $\mathcal{D}$ looks li,ke this
\begin{equation}
\log p (\mathcal{D}|\bstheta) = \sum\limits_{n=1}^{N}\log p(\x|\bstheta) = \sum_{k=1}^{K}\pi_k\prod_{j=1}^{D}\Cat(x_j|\bstheta_k).
\end{equation}
However this function is hard to optimize due to the sum inside the $\log$. In order to solve this we introduce the latent variable $\mathcal{Z}$. The marginal distribution for the hidden variable $z_i$ is defined by the mixing coefficient of the mixture $\pi_k$ as follows
\begin{equation}
p(z_i=k) = \pi_k.
\end{equation}
Since $\bspi = \{\pi_1,\dots,\pi_K\}$ is a vector with the probabilities for the mixtures, certain restrictions must be satisfied:
\begin{align}\label{eq:pi_constraint}
& 0 < \pi_k < 1 \\
\text{and} \quad  & \sum\limits_{k=1}^{K} \pi_k = 1.
\end{align}
Now we need to find the expression for the complete data log li,kelihood $p(\mathcal{D},\mathcal{Z})$. First of all, the marginal of $z_i$ is given by
\begin{equation}\label{eq:marginal_z}
p(z_i|\bstheta) = \prod_{k=1}^{K}\pi_k^{\Ind \lbrace z_i=k \rbrace},
\end{equation}
where $\Ind(\cdot)$ is the indicator function. The probability density function for $x_i$ given $z_i$ is given by the following expression
\begin{equation}\label{eq:x_given_z}
p(\x_i|z_i,\bstheta) = \prod_{k=1}^{K}\prod_{j=1}^{D}\Cat(x_{ij}|\bstheta_k)^{\Ind \lbrace z_i=k \rbrace}.
\end{equation}
Since we are interested in the joint probability, applying Bayes' rule
\begin{equation}
p\left(\x_{i},z_{i}|\bstheta\right) = p\left(\x_{i}|z_{i},\bstheta\right) p(z_{i}|\bstheta)
\end{equation}
yields the resulting expression
\begin{equation}
p\left(\x_{i},z_{i}|\bstheta\right) = \prod\limits_{k=1}^{K}\left( \pi_{k} \prod \limits_{j=1}^{D} \text{Cat}(x_{ij}|\bstheta_{k}) \right)^{\Ind\lbrace z_{i}=k \rbrace}.
\end{equation}

\section{Complete data log li,kelihood $l_c(\bstheta) $}
\noindent Assuming that our observed data is independent and identically distributed (i.i.d) and taking natural logarithm we can find expression of the complete data log li,kelihood
\begin{align}
l_c(\bstheta) & = \log p(\mathcal{D},\mathcal{Z}|\bstheta) = \log \prod\limits_{n=1}^{N} p\left(\x_{i},z_{i}|\bstheta\right)\\
& = \sum_{i=1}^{N}\log p\left(\x_{i},z_{i}|\bstheta\right)\\
& = \sum_{i=1}^{N} \log \prod\limits_{k=1}^{K}\left( \pi_{k} \prod \limits_{j=1}^{D} \text{Cat}(x_{ij}|\bstheta_{k}) \right)^{\Ind\lbrace z_{i}=k \rbrace}\\
& = \sum_{i=1}^N\sum_{k=1}^K\Ind (z_i=k) \log\left( \pi_k\prod_{j=1}^D\Cat(x_{ij}|\bstheta_k)\right).
\end{align}
We can simplify the expression for the marginal of $\x_i$ as
\begin{align}\label{eq:marginal_x}
    p(\x_i) & = \prod_{j=1}^D\Cat(x_{ij}|\bstheta) = \prod_{j=1}^D\prod_{m=1}^I\theta_m^{\Ind \lbrace x_{i,j}=m \rbrace} = \prod_{m=1}^I\prod_{j=1}^D\theta_m^{\Ind \lbrace x_{i,j}=m \rbrace } \\
    & = \prod_{m=1}^I\theta_m^{\sum_{j=1}^D {\Ind \lbrace x_{i,j}=m \rbrace }} = \prod_{m=1}^I \theta_m^{\mu_{i,m}},
\end{align}
where we have defined a new metric $\mu_{i,m}$ that represents the number of times the word associated to the category $m$ appears at document $i$, \textit{i.e}.\ ,
\begin{equation}
    \mu_{i,m} = \sum_{j=1}^D {\Ind(x_{i,j}=m)}.
\end{equation}
Therefore, the final expression will be 
\begin{align}
    l_c(\bstheta) & = \sum_{i=1}^N\sum_{k=1}^K\Ind (z_i=k) \log\left( \pi_k\prod_{j=1}^D\Cat(x_{ij}|\bstheta_k)\right) \\
    & =  \sum_{i=1}^N\sum_{k=1}^K\Ind (z_i=k) \log \pi_k + \sum_{i=1}^N\sum_{k=1}^K\Ind (z_i=k) \log \prod_{m=1}^I \theta_{k,m}^{\mu_{i,m}}\\
    & = \sum_{i=1}^N\sum_{k=1}^K\Ind (z_i=k) \log \pi_k + \sum_{i=1}^N\sum_{k=1}^K\Ind (z_i=k) \sum_{m=1}^I \mu_{i,m}\log \theta_{k,m} \label{eq:log_li,ke}.
\end{align}
    
   
\section{ML Inference}

\noindent In order to solve the maximum li,kelihood problem, we define an auxiliary function $Q$ which will be the expected complete data log li,kelihood over the hidden variable $\mathcal{Z}$. This function $Q$ will be evaluated in the E-step and maximized for the model parameters $\bstheta$ to update the model parameters in each iteration of the algorithm.
\subsection{E-step: $Q(\bstheta^t, \bstheta^{t-1})$ for ML inference}

Taking the expectation with respect to the latent variables $z$ we get the following expected complete data log-li,kelihood.
\begin{align}
    Q(\bstheta^t, \bstheta^{t-1}) & = \expectation_Z\{l_c(\bstheta)|\mathcal{D},\bstheta^{t-1}\}  \\
    & = \sum_{i=1}^N\sum_{k=1}^K\expectation_Z\{\Ind (z_i=k)\} \log \pi_k + \sum_{i=1}^N\sum_{k=1}^K\expectation_Z\{\Ind(z_i=k)\} \sum_{m=1}^I \mu_{i,m}\log \theta_{k,m}\\
    & = \sum_{i=1}^N\sum_{k=1}^K r_{i,k}\log \pi_k + \sum_{i=1}^N\sum_{k=1}^K r_{i,k}\sum_{m=1}^I \mu_{i,m}\log \theta_{k,m} \label{eq:Q_function},
\end{align}
where we have defined the metric $r_{i,k} \triangleq p(z_i = k | \x_i,\bstheta^{t-1})$ \cite{murphy2012machine} as the responsibility that mixture $k$, \textit{i.e.\ } topic $k$, takes at explaining the document $\x_i$. It is defined as follows
\begin{align}\label{eq:r_i,k}
	r_{i,k} & = \expectation_Z\{ \Ind(z_i=k) \} = p(z_i = k|\x_i,\bstheta^{t-1}) \\
	& = \frac{p(z_i=k,\x_i|\bstheta^{t-1})}{p(\x_i|\bstheta^{t-1})} = \frac{p(z_i=k,\x_i|\bstheta^{t-1})}{\sum \limits_{k\prime}^{K}p(z_i=k\prime,\x_i|\bstheta^{t-1})}\\
	& = \frac{\pi_k p(\x_i|\bstheta_k)}{\sum \limits_{k\prime}^K \pi_{k\prime} p(\x_i|\bstheta_{k\prime})}  =	\frac{\pi_k\prod_{j=1}^{D}\Cat(x_{i,j}|\bstheta_k)}{\sum \limits_{k\prime}^{K} \pi_{k\prime} \prod \limits_{j=1}^{D}\Cat(x_{i,j}|\bstheta_k\prime)}.
\end{align}
This quantity must satisfy the following constrains, given that z it is a probability
\begin{align}\label{eq:ri,k_constriant}
& 0 \leq r_{i,k} \leq 1, \\
\text{and} \quad & \sum \limits_{k=1}^{K} r_{i,k} = 1.
\end{align}

\subsubsection{Arithmetic underflow solved by log-sum-exp trick}

In our problem we have $I$ categories which correspond to the words from a dictionary. When we perform the product in Equation \ref{eq:r_i,k} we obtain really small values that cannot be represented by the computer. This problem is known as \textbf{arithmetic underflow}. However, in order to maintain the range in probability if categories $\theta_{k,m}$ that influence the value of $r_{i,k}$ we use the \textbf{log-sum-exp trick} \cite{logsumexp}.

\noindent Taking the logarithm from Equation \ref{eq:r_i,k} and using Equation \ref{eq:marginal_x} for $p(\x_i|\bstheta_k)$ we obtain the following expression
\begin{align}\label{eq:log_rik}
\log r_{i,k} & = \log \pi_k\prod_{j=1}^{D}\Cat(x_{i,j}|\bstheta_k) - \log \sum \limits_{k\prime=1}^{K} \pi_{k\prime} \prod \limits_{j=1}^{D}\Cat(x_{i,j}|\bstheta_k\prime)\\
& = \log \pi_k + \sum\limits_{m=1}^{I}\mu_{i,m}\log\theta_{k,m} - \log \sum \limits_{k\prime=1}^{K} \pi_{k\prime} \prod \limits_{m=1}^{I} \theta_{k\prime,m}^{\mu_{i,m}}.
\end{align}
The log-sum-exp trick will be applied on the last term from the above equation. The trick states that
\begin{equation}
\log \sum _ { v = 1 } ^ { V } e ^ { g _ { v } } = a + \log \sum _ { v = 1 } ^ { V } e ^ { g _ { v } - a },
\end{equation}
where $a = \max_{\substack{v}} g_v$ and $g_v$ is defined as 
\begin{equation}
g_v = \log \pi_{k\prime} \prod \limits_{m=1}^{I} \theta_{k\prime,m}^{\mu_{i,m}} = \log \pi_{k\prime} + \sum\limits_{m=1}^{I}\mu_{i,m} \log \theta_{k\prime,m}-
\end{equation}
Once $\log r_{i,k}$ from Equation \ref{eq:log_rik} is known, we can find $r_{i,k}$ with an exponential. With this trick we do not loose any information by forcing values assigned to be 0 by the computer using clipping and we also increase computational speed. 

\subsection{M-step for ML inference}

\noindent Now we need to find the closed-form formulas to update the model parameters $\bstheta = \{\bstheta_k, \bspi\}$. Since we must tackle a maximization problem with constraints we will apply Lagrange Multipliers to solve equation
\begin{equation}
\boldsymbol { \theta } ^ { t } = \underset { \boldsymbol { \theta } } { \operatorname { argmax } } Q \left( \boldsymbol { \theta } , \boldsymbol { \theta } ^ { t - 1 } \right).
\end{equation}

\subsubsection{Maximization of $\pi_k$}

\noindent Using the constraints on $\pi$ from Equations \ref{eq:pi_constraint} we propose as Lagrangian the function
\begin{equation}\label{eq:lagrange_pi,k}
L\left( Q(\pi_{k}),\lambda \right) = Q(\pi_{k}) + \lambda \left( \sum \limits_{k=1}^{K} \pi_{k} - 1 \right),
\end{equation}
which will be optimized in this way
\begin{equation}\label{minmax_lagrange_pi,k}
\min_{\substack{\lambda}}\max_{\substack{\pi_{k}}} \lbrace L\left( Q(\pi_{k}),\lambda \right) \rbrace.
\end{equation}

\noindent  First, we take the derivative of Equation \ref{eq:lagrange_pi,k} w.r.t $\pi_{k}$ and equate it to 0, which yields
\begin{align}\label{deriv_pi,k}
& \dfrac{\partial L}{\partial \pi_{k}} = 0 = \sum \limits_{i=1}^{N} \dfrac{r_{i,k}}{\pi_{k}} - \lambda, \\
& \pi_{k} = \dfrac{1}{\lambda} \sum \limits_{i=1}^{N} r_{i,k} \label{eq:pi_k_deriv}.
\end{align}
Now, we take the derivate w.r.t $\lambda$ which yields
\begin{align}
& \dfrac{\partial L}{\partial \lambda} = \sum \limits_{k=1}^{K} \pi_{k} - 1  = 0 ,\\
& \sum \limits_{k=1}^{K} \pi_{k} = 1.
\end{align}
If we sum over $k$ at both sides of Equation \ref{eq:pi_k_deriv}
\begin{align}
& \sum \limits_{k=1}^{K}\pi_{k} = \sum \limits_{k=1}^{K} \dfrac{1}{\lambda} \sum \limits_{i=1}^{N} r_{i,k}\\
& 1 = \dfrac{1}{\lambda} \sum \limits_{i=1}^{N}\sum \limits_{k=1}^{K}r_{i,k},
\end{align}
and using the constraints on $r_{i,k}$ from Equations \ref{eq:ri,k_constriant} we get the value of $\lambda$
\begin{align}
& 1 = \dfrac{1}{\lambda} \sum \limits_{i=1}^{N}1.\\
& \lambda = N.
\end{align}
Knowing the value of $\lambda$ the estimated value of $\hat{\pi}_{k}$ can be expressed as follows
\begin{align}\label{eq:pi_k}
& \hat{\pi}_{k} = \dfrac{1}{N} \sum \limits_{i=1}^{N} r_{i,k}s = \frac{N_k}{N},
\end{align}
where 
\begin{equation}\label{eq:N_k}
N_k = \sum_{i=1}^N r_{i,k}.
\end{equation}
This result is actually intuitive since $N_k$ represents the 'weight' of topic $k$ at explaining the documents, and therefore $\pi_k$ is just the percentage of topic $k$ at explaining the data.

\subsubsection{Maximization of $\bstheta_k$}

\noindent We follow a similar approach. Now, using the constraints on $\bstheta_k$ from Equations \ref{eq:theta_constraint} we propose as Lagrangian the following function
\begin{equation}\label{eq:lagrange_thetakm}
L\left( Q(\theta_{k,m}),\lambda \right) = Q(\theta_{k,m}) + \lambda \left( \sum \limits_{m=1}^{I} \theta_{k,m} - 1 \right),
\end{equation}
which will be optimized as a min-max problem
\begin{equation}\label{minmax_lagrange_thetakm}
\min_{\substack{\lambda}} \max_{\substack{\theta_{km}}} \lbrace L\left( Q(\theta_{km}),\lambda \right) \rbrace.
\end{equation}
Firstly, we take the derivative of Equation \ref{eq:lagrange_thetakm} w.r.t $\theta_{k,m}$ and equate it to 0, which yields
\begin{align}\label{deriv_thetakm}
& \dfrac{\partial L}{\partial \theta_{k,m}} = \sum \limits_{i=1}^{N} \dfrac{r_{i,k}\mu_{i,m}}{\theta_{k,m}} - \lambda = 0 , \\
& \theta_{k,m} = \dfrac{1}{\lambda} \sum \limits_{i=1}^{N} r_{i,k}\mu_{i,m} \label{eq:theta_k_deriv}.
\end{align}
Secondly, we take the derivative w.r.t $\lambda$, which yields
\begin{align}
& \dfrac{\partial L}{\partial \lambda} = \sum \limits_{m=1}^{I} \theta_{k,m} - 1 = 0,\\
& \sum \limits_{m=1}^{I} \theta_{k,m} = 1.
\end{align}
If now we sum over $I$ at both sides of Equation \ref{eq:theta_k_deriv}, the value of $\lambda$ is obtained
\begin{align}
& \sum \limits_{m=1}^{I}\theta_{k,m} = \sum \limits_{m=1}^{I} \dfrac{1}{\lambda} \sum \limits_{i=1}^{N} r_{i,k}\mu_{i,m}\\
& 1 = \dfrac{1}{\lambda} \sum \limits_{i=1}^{N}\sum \limits_{m=1}^{I}r_{i,k}\mu_{i,m},\\
& \lambda = \sum \limits_{i=1}^{N}\sum \limits_{m=1}^{I}r_{i,k}\mu_{i,m}.
\end{align}
Using the value of $\lambda$ in Equation \ref{eq:theta_k_deriv}, $\theta_{k,m}$ is obtained
\begin{equation}\label{eq:theta_km}
\hat{\theta}_{k,m} = \dfrac{\sum \limits_{i=1}^{N} r_{i,k}\mu_{i,m}}{\sum \limits_{i=1}^{N}\sum \limits_{m=1}^{I}r_{i,k}\mu_{i,m}}.
\end{equation}
Again, the result is quite intuitive. $\theta_{k,m}$ is just the average of the category $m$ weighted by the responsibility $r_{i,k}$.

\section{MAP inference}

Maximum li,kelihood estimations is an estimation that tends to overfit \cite{murphy2012machine}. A solution to such problem is applying maximum a posteriori estimation (MAP). In this case, we do not only consider the li,kelihood, but also some prior information on the model parameters $\bstheta$. From Bayes' rule we know that
\begin{equation}
p(\boldsymbol{\theta}|\mathcal{D},\mathcal{Z}) \varpropto p(\mathcal{D},\mathcal{Z}|\boldsymbol{\theta})p(\boldsymbol{\theta}).
\end{equation}
If we take logarithms at both sides of the equation we get
\begin{equation}
\log p(\boldsymbol{\theta}|\mathcal{D},\mathcal{Z}) \varpropto \log p(\mathcal{D},\mathcal{Z}|\boldsymbol{\theta}) + \log p(\boldsymbol{\theta}) .
\end{equation}
Notice that the first term is just the complete data log li,kelihood from Equation \ref{eq:log_li,ke} and the second term corresponds to the prior information. The goal now is to define some prior over the model parameters $\bstheta$ and use it on the function $Q$ from Equation \ref{eq:Q_function} to work with the posterior instead of the li,kelihood. Afterwards, we need to reformulate Equation \ref{eq:pi_k} and \ref{eq:theta_km} for $\pi_k$ and $\theta_{k,m}$ to take into account the priors.\\

\subsection{E-step: $Q(\bstheta^t, \bstheta^{t-1})$ for MAP inference}
\noindent From \cite{notesArtes} and \cite{murphy2012machine} we know it is natural that the prior on the mixture weights $\pi$ and the category probabilities $\bstheta_k$ follow a Dirichlet distribution, \textit{i.e.\ },
\begin{align}\label{priors}
& \bspi \sim \text{Dir}(\bsbeta), \quad \text{s.t.} \quad  p(\bspi |\bsbeta) = \dfrac{1}{\text{B}(\boldsymbol{\beta})} \prod \limits_{k=1}^{K} \pi_{k}^{\beta_{k}-1}, \\
& \bstheta_{k} \sim \text{Dir}(\bsalpha), \quad \text{s.t.} \quad p(\bstheta_{k}|\bsalpha) = \dfrac{1}{\text{B}(\boldsymbol{\alpha})} \prod \limits_{m=1}^{I} \theta_{k,m}^{\alpha_{m}-1},
\end{align}
where the function B(·) stands for the Beta function, and the parameters $\boldsymbol{\beta}$ and $\boldsymbol{\alpha}$ are the parameters or hyperpriors of the Dirichlet distributions for $\bspi$ and $\bstheta_{k}$ respectively. \\


\noindent The expression for the new $Q$ function is as follows
\begin{equation}\label{eq:Q_map_deriv}
Q(\bstheta^t, \bstheta^{t-1}) = \sum_{i=1}^N\sum_{k=1}^K r_{i,k}\log \pi_k + \sum_{i=1}^N\sum_{k=1}^K r_{i,k}\sum_{m=1}^I \mu_{i,m}\log \theta_{k,m} + \log p(\bspi|\bsbeta) + \sum\limits_{k=1}^{K}\log p(\bstheta_k|\bsalpha),
\end{equation}
where $r_{i,k}$ remains the same. Taking logarithm on the probability of the priors we get
\begin{align}
& \log p(\bspi|\bsbeta) = \log \left (\dfrac{1}{\text{B}(\bsbeta)} \prod \limits_{k=1}^{K} \pi_{k}^{\beta_{k}-1} \right )= \log \dfrac{1}{\text{B}(\bsbeta)} + \sum \limits_{k=1}^{K}(\beta_{k}-1)\text{log}(\pi_{k}),\\
& \log p(\bstheta_{k}|\bsalpha) = \log \left(\dfrac{1}{\text{B}(\bsalpha)} \prod \limits_{m=1}^{I} \theta_{k,m}^{\alpha_{m}-1}\right) = \log \dfrac{1}{\text{B}(\bsalpha)} +  \sum \limits_{m=1}^{I}(\alpha_{m}-1)\log (\theta_{k,m}).
\end{align}
\noindent Using these results in Equation \ref{eq:Q_map_deriv} we get the expression for $Q(\bstheta,\bstheta^{t-1})$ using MAP estimation
\begin{equation}\label{eq:Q_map}
\begin{split}
Q(\bstheta^t, \bstheta^{t-1}) = & \sum_{i=1}^N\sum_{k=1}^K r_{i,k}\log \pi_k + \sum_{i=1}^N\sum_{k=1}^K r_{i,k}\sum_{m=1}^I \mu_{i,m}\log \theta_{k,m} \\
& + \log \left(\dfrac{1}{\text{B}(\bsbeta)}\right) + \sum \limits_{k=1}^{K}(\beta_{k}-1)\text{log}(\pi_{k})\\
 & + \sum\limits_{k=1}^{K} \log \dfrac{1}{\text{B}(\bsalpha)} +  \sum\limits_{k=1}^{K}\sum \limits_{m=1}^{I}(\alpha_{m}-1)\log (\theta_{k,m}).
\end{split}
\end{equation}
Notice that $\text{B}(\bsalpha)$ can be decomposed as $$
\mathrm { B } ( \boldsymbol { \alpha } ) = \frac { \prod _ { i = 1 } ^ { K } \Gamma \left( \alpha _ { i } \right) } { \Gamma \left( \sum _ { i = 1 } ^ { K } \alpha _ { i } \right) },
$$
where $\Gamma(\cdot)$ is the Gamma function. The same applies for $\text{B}(\bsbeta)$


\subsection{M step for MAP inference}

\noindent We will follow the same procedure as for the ML case; but now we must use function \ref{eq:Q_map} instead to take into account the priors.

\subsubsection{MAP estimation of $\pi_{k}$}

\noindent Again, we have the same maximization problem and hence we propose the same Lagrangian as in Equation \ref{eq:lagrange_pi,k} and we use the same constraints on $\pi_k$
\begin{equation}
L\left( Q(\pi_{k}),\lambda \right) = Q(\pi_{k}) + \lambda \left( \sum \limits_{k=1}^{K} \pi_{k} - 1 \right).
\end{equation}
Again, we derivative Equation \ref{eq:Q_map} w.r.t $\pi_{k}$ and equate it to 0
\begin{align}
& \dfrac{\partial L}{\partial \pi_{k}} =  \sum \limits_{i=1}^{N} \dfrac{r_{i,k}}{\pi_{k}} + \dfrac{\beta_{k}-1}{\pi_k} - \lambda = 0 ,\\
& \pi_{k} = \dfrac{1}{\lambda} \left( \sum \limits_{i=1}^{N} r_{i,k} + \beta_{k} - 1\right)\label{eq:deriv_pi,k_MAP}.
\end{align}
Now we derivate w.r.t $\lambda$
\begin{align}
& \dfrac{\partial L}{\partial \lambda} = \sum \limits_{k=1}^{K} \pi_{k} - 1 = 0 , \\
& \sum \limits_{k=1}^{K} \pi_{k} = 1.
\end{align}
Summing over $K$ at both sides of Equation \ref{eq:deriv_pi,k_MAP}
\begin{align}
& \sum \limits_{k=1}^{K}\pi_{k} = \sum \limits_{k=1}^{K} \dfrac{1}{\lambda} \left( \sum \limits_{i=1}^{N} r_{i,k} + \beta_{k} - 1\right)\\
& 1 = \dfrac{1}{\lambda} \sum \limits_{k=1}^{K} \sum \limits_{i=1}^{N} r_{i,k} + \dfrac{1}{\lambda} \sum \limits_{k=1}^{K} \beta_{k} - \dfrac{1}{\lambda} \sum \limits_{k=1}^{K} 1,
\end{align}
we get the value of $\lambda$
\begin{equation}
\lambda = N + \sum \limits_{k=1}^{K} \beta_{k} - K.
\end{equation}
Once $\lambda$ is known the estimated value of $\pi_{k}$ is as follows
\begin{equation}
\hat{\pi}_{k} = \dfrac{\sum \limits_{i=1}^{N} r_{i,k} + \beta_{k} - 1}{N + \sum \limits_{k=1}^{K} \beta_{k} - K}.
\end{equation}

\subsubsection{MAP estimation of $\theta_{k,m}$}

\noindent For the estimation of $\theta_{k,m}$ we use the same Lagrangian from Equation \ref{eq:lagrange_thetakm} and the same restrictions from \ref{eq:theta_constraint} 
\begin{equation}
L\left( Q(\theta_{k,m}),\lambda \right) = Q(\theta_{k,m}) + \lambda \left( \sum \limits_{m=1}^{I} \theta_{k,m} - 1 \right).
\end{equation}
As before, we first derivate w.r.t $\theta_{k,m}$, which yields
\begin{align}
& \dfrac{\partial L}{\partial \theta_{k,m}} = \sum \limits_{i=1}^{N} \dfrac{r_{i,k}\mu_{im}}{\theta_{k,m}} + \dfrac{\alpha_m-1}{\theta_{km}} - \lambda = 0, \\
& \theta_{k,m} = \dfrac{1}{\lambda} \left( \sum \limits_{i=1}^{N} r_{i,k}\mu_{i,m} +\alpha_m - 1\right) \label{eq:deriv_thetakm_MAP}.
\end{align}
And later with respect to $\lambda$
\begin{align}
& \dfrac{\partial L}{\partial \lambda} = 0 = \sum \limits_{m=1}^{I} \theta_{k,m} - 1,\\
& \sum \limits_{m=1}^{I} \theta_{k,m} = 1.
\end{align}
Summing at both sides over $I$ Equation \ref{eq:deriv_thetakm_MAP} we can obtain the value of $\lambda$
\begin{align}
& \sum \limits_{m=1}^{I}\theta_{k,m} = \sum \limits_{m=1}^{I} \dfrac{1}{\lambda} \left( \sum \limits_{i=1}^{N} r_{i,k}\mu_{i,m} + \alpha_m - 1 \right)\\
& 1 = \dfrac{1}{\lambda} \sum \limits_{m=1}^{I} \left( \sum \limits_{i=1}^{N} \dfrac{r_{i,k}\mu_{i,m}}{\theta_{k,m}} + \dfrac{\alpha_m}{\theta_{k,m}} - 1 \right),\\
& \lambda = \sum \limits_{m=1}^{I} \sum \limits_{i=1}^{N} r_{i,k}\mu_{i,m} + \sum \limits_{m=1}^{I} \alpha_m - \sum \limits_{m=1}^{I} 1,\\
& \lambda = \sum \limits_{m=1}^{I} \sum \limits_{i=1}^{N} r_{i,k}\mu_{i,m} + \sum \limits_{m=1}^{I} \alpha_m - I.
\end{align}
And with that value of $\lambda$ the estimated value of $\theta_{k,m}$ is
\begin{equation}
\hat{\theta}_{k,m} = \dfrac{\sum \limits_{i=1}^{N} r_{i,k}\mu_{im} +\alpha_m - 1}{\sum \limits_{m=1}^{I} \sum \limits_{i=1}^{N} r_{i,k}\mu_{i,m} + \sum \limits_{m=1}^{I} \alpha_m - I}.
\end{equation}

\section{Experiments}

\subsection{ML Experiments}
INCLUDE FIGURES
\subsection{MAP Experiments}
INCLUDE FIGURES
\nocite{*}
\printbibliography

\end{document}

